\documentclass[11pt, oneside]{article}   	% use "amsart" instead of "article" for AMSLaTeX format
\usepackage{geometry}                		% See geometry.pdf to learn the layout options. There are lots.
\geometry{letterpaper}                   		% ... or a4paper or a5paper or ... 
%\geometry{landscape}                		% Activate for rotated page geometry
%\usepackage[parfill]{parskip}    		% Activate to begin paragraphs with an empty line rather than an indent
\usepackage{graphicx}				% Use pdf, png, jpg, or eps§ with pdflatex; use eps in DVI mode
								% TeX will automatically convert eps --> pdf in pdflatex		
\usepackage{amssymb}

%SetFonts

%SetFonts


\title{Brief Article}
\author{The Author}
%\date{}							% Activate to display a given date or no date

\begin{document}
\maketitle
%\section{}
%\subsection{}
The IS equations reads,

\begin{equation}
D\pi^{\langle \mu \nu \rangle} = - \frac{\pi^{\mu\nu} - \pi_{NS}^{\mu\nu}}{\tau_{\pi}} - \frac{4}{3}\pi^{\mu\nu}\theta \end{equation}


Where $A^{\langle \mu \nu \rangle} = \Delta^{\mu\nu\alpha\beta} A_{\alpha \beta}$ and $D \equiv u^{\lambda} \partial_{;\lambda}$ where the $\partial_{;\lambda}$ is the corviant differential. The expansion rate is $\theta=\partial_{;\lambda}u^{\lambda}$. None zero Christoeffel terms are $\Gamma^{\eta}_{\eta\tau}= \Gamma^{\eta}_{\tau\eta}= \frac{1}{\tau},\ \Gamma^{\tau}_{\eta\eta}=\tau$.

$\pi_{NS}^{\mu\nu} = \eta_v \sigma^{\mu\nu}$ where $\sigma^{\mu\nu} = \partial_{;}^{<\mu}u^{\nu>} = 2 \Delta^{\mu\nu\alpha\beta}\partial_{;\alpha}u_{\beta} $.

The projectors used here are,

\begin{eqnarray}
\Delta^{\mu\nu} &=& g^{\mu\nu} - u^{\mu} u^{\nu} \\
\Delta^{\mu\nu\alpha\beta} &=& \frac{1}{2}(\Delta^{\mu\alpha} \Delta^{\nu\beta} + \Delta^{\nu\alpha} \Delta^{\mu\beta}) - \frac{1}{3}\Delta^{\mu\alpha} \Delta^{\nu\beta}
\end{eqnarray}


The aim of this note is to use sympy to get all the christoeffel terms and move them to the right hand side.
The upper equation is rewritten as,


\begin{eqnarray}
\Delta^{\mu\nu\alpha\beta} D \pi_{\alpha \beta} &=& - \frac{\pi^{\mu\nu} - \pi_{NS}^{\mu\nu}}{\tau_{\pi}} - \frac{4}{3}\pi^{\mu\nu}\theta\\
\Delta^{\mu\nu\alpha\beta} u^{\lambda}\partial_{\lambda}\pi_{\alpha \beta} &=& f + u^{\lambda}\Delta^{\mu\nu\alpha\beta}(\Gamma^{\rho}_{\alpha\lambda}\pi_{\rho\beta} - \Gamma^{\rho}_{\beta\lambda}\pi_{\alpha\rho})
\end{eqnarray}


The left hand side of the equation reads,
\begin{eqnarray}
\Delta^{\mu\nu\alpha\beta} D \pi_{\alpha \beta} &=& \left[ \frac{1}{2}(\Delta^{\mu\alpha} \Delta^{\nu\beta} + \Delta^{\nu\alpha} \Delta^{\mu\beta}) - \frac{1}{3}\Delta^{\mu\alpha} \Delta^{\nu\beta} \right] D\pi_{\alpha \beta} \\
&=& \frac{1}{2} \Delta^{\mu\alpha} \Delta^{\nu\beta} (D\pi_{\alpha \beta} + D\pi_{\beta\alpha}) - \frac{1}{3}\Delta^{\mu\nu} \Delta^{\alpha\beta} D\pi_{\alpha \beta} \\
&=& \Delta^{\mu\alpha} \Delta^{\nu\beta} D\pi_{\alpha \beta} - \frac{1}{3}\Delta^{\mu\nu} \left( D(\Delta^{\alpha\beta} \pi_{\alpha \beta}) - \pi_{\alpha \beta} D\Delta^{\alpha\beta}\right) \\
&=& (g^{\mu\alpha} - u^{\mu}u^{\alpha})(D\pi_{\alpha}^{\nu} + u^{\nu}\pi_{\alpha\beta}Du^{\beta}) + \frac{1}{3}\Delta^{\mu\nu} \pi_{\alpha \beta} D\Delta^{\alpha\beta} \\
&=& D\pi^{\mu\nu} + u^{\nu}\pi^{\mu\beta} Du_{\beta} + u^{\mu} \pi^{\nu\beta}Du_{\beta} + 0 \\
&=& D\pi^{\mu\nu} + (u^{\nu}\pi^{\mu\beta} + u^{\mu} \pi^{\nu\beta})Du_{\beta}
\end{eqnarray}
where $Dg^{\alpha\beta}=0$.

The IS equation can be rewritten as,
\begin{equation}
D\pi^{\mu\nu} = - \frac{\pi^{\mu\nu} - \pi_{NS}^{\mu\nu}}{\tau_{\pi}} - \frac{4}{3}\pi^{\mu\nu}\theta - g_{\alpha\beta}(u^{\nu}\pi^{\mu\beta} + u^{\mu} \pi^{\nu\beta})Du^{\alpha}
\end{equation}
where the left hand side can be expanded as,

\begin{eqnarray}
D\pi^{\mu\nu} &=& u^{\lambda}\partial_{;\lambda}\pi^{\mu\nu} \\
&=& u^{\lambda}\left( \partial_{\lambda}\pi^{\mu\nu} + \Gamma^{\mu}_{\alpha\lambda}\pi^{\alpha\nu} + \Gamma^{\nu}_{\alpha\lambda}\pi^{\mu\alpha}\right) \\
&=& u^{\lambda}\partial_{\lambda}\pi^{\mu\nu}  + u^{\tau}\left(\Gamma^{\mu}_{\alpha\tau}\pi^{\alpha\nu} + \Gamma^{\nu}_{\alpha\tau}\pi^{\mu\alpha}\right) + u^{\eta}\left(\Gamma^{\mu}_{\alpha\eta}\pi^{\alpha\nu} + \Gamma^{\nu}_{\alpha\eta}\pi^{\mu\alpha}\right) \\
&=& \partial_{\lambda}(u^{\lambda}\pi^{\mu\nu})  + u^{\tau}\left(\Gamma^{\mu}_{\alpha\tau}\pi^{\alpha\nu} + \Gamma^{\nu}_{\alpha\tau}\pi^{\mu\alpha}\right) + u^{\eta}\left(\Gamma^{\mu}_{\alpha\eta}\pi^{\alpha\nu} + \Gamma^{\nu}_{\alpha\eta}\pi^{\mu\alpha}\right) - \pi^{\mu\nu}\theta + \pi^{\mu\nu}\frac{u^{\tau}}{\tau}
\end{eqnarray}

Finally the IS equations are simplified to:
\begin{equation}
\partial_{\lambda}(u^{\lambda}\pi^{\mu\nu}) =  - \frac{\pi^{\mu\nu} - \pi_{NS}^{\mu\nu}}{\tau_{\pi}} - \frac{1}{3}\pi^{\mu\nu}\theta - g_{\alpha\beta}(u^{\nu}\pi^{\mu\beta} + u^{\mu} \pi^{\nu\beta})Du^{\alpha}  + \pi^{\mu\nu}\frac{u^{\tau}}{\tau} - Christ
\end{equation}

where the Christoffel terms are,
$Christ = u^{\tau}\left(\Gamma^{\mu}_{\alpha\tau}\pi^{\alpha\nu} + \Gamma^{\nu}_{\alpha\tau}\pi^{\mu\alpha}\right) + u^{\eta}\left(\Gamma^{\mu}_{\alpha\eta}\pi^{\alpha\nu} + \Gamma^{\nu}_{\alpha\eta}\pi^{\mu\alpha}\right) $.

Now we want to do some substitution with,

\begin{eqnarray}
    \tilde{g}^{\mu\nu} &=& \tilde{g}_{\mu\nu} = diag(1, -1, -1, -1) \\
    u^{\mu} &=& (\tilde{u}^{\tau}, \tilde{u}^{x}, \tilde{u}^{y}, \tilde{u}^{\eta}/\tau) \\
    \partial_{\lambda} &=& (\tilde{\partial}_{\tau}, \tilde{\partial}_{x}, \tilde{\partial}_{y}, \tilde{\partial}_{\eta}/\tau) \\
    \pi^{\mu\nu} &=& \tilde{\pi}^{\mu\nu}  \   for\ \mu \neq \eta \ and \ \nu \neq \eta  \\
    \pi^{\mu\eta} &=& \tilde{\pi}^{\mu\eta}/\tau  \   for\ \mu \neq \eta \\
    \pi^{\eta\eta} &=& \tilde{\pi}^{\eta\eta}/\tau^2 \\
\end{eqnarray}

Then we have the following unchanged forms,
\begin{eqnarray}
    D &=& \tilde{D} \\
    \theta &=& \tilde{\theta} \\
    g_{\alpha\beta}(u^{\nu}\pi^{\mu\beta} + u^{\mu} \pi^{\nu\beta})Du^{\alpha} &=& \tilde{g}_{\alpha\beta}(\tilde{u}^{\nu}\tilde{\pi}^{\mu\beta} + \tilde{u}^{\mu} \tilde{\pi}^{\nu\beta})\tilde{D}\tilde{u}^{\alpha}
\end{eqnarray}

And the left hand side of IS equation reads,
\begin{eqnarray}
\partial_{\lambda}(u^{\lambda}\pi^{\mu\nu}) &=& \tilde{\partial}_{\lambda}(\tilde{u}^{\lambda}\tilde{\pi}^{\mu\nu})  \ for \ \mu \neq \eta \ and \ \nu \neq \eta \\
\partial_{\lambda}(u^{\lambda}\pi^{\mu\eta}) &=& \partial_{\lambda}(u^{\lambda}\tilde{\pi}^{\mu\eta}/\tau) = \frac{1}{\tau}\tilde{\partial}_{\lambda}(\tilde{u}^{\lambda}\tilde{\pi}^{\mu\eta}) - \frac{\tilde{u}^{\tau}\tilde{\pi}^{\mu\eta}}{\tau^2}  \ for \ \mu \neq \eta \\
\partial_{\lambda}(u^{\lambda}\pi^{\eta\eta}) &=& \partial_{\lambda}(u^{\lambda}\tilde{\pi}^{\eta\eta}/\tau^2) = \frac{1}{\tau^2}\tilde{\partial}_{\lambda}(\tilde{u}^{\lambda}\tilde{\pi}^{\eta\eta}) - \frac{2\tilde{u}^{\tau}\tilde{\pi}^{\eta\eta}}{\tau^3} \\
\end{eqnarray}

For the $\pi^{\eta\eta}$ term we now have the IS equation, together with Christoffel terms given below,

\begin{eqnarray}
\frac{1}{\tau^2}\tilde{\partial}_{\lambda}(\tilde{u}^{\lambda}\tilde{\pi}^{\eta\eta}) - \frac{2\tilde{u}^{\tau}\tilde{\pi}^{\eta\eta}}{\tau^3} =   - \frac{\tilde{\pi}^{\eta\eta}/\tau^2 - \tilde{\pi}_{NS}^{\eta\eta}/\tau^2}{\tau_{\pi}} - \frac{1}{3}\tilde{\pi}^{\eta\eta}\tilde{\theta}/\tau^2 - \tilde{g}_{\alpha\beta}(\tilde{u}^{\eta}\tilde{\pi}^{\eta\beta} + \tilde{u}^{\eta} \tilde{\pi}^{\eta\beta})\tilde{D}\tilde{u}^{\alpha} /\tau^2 + \tilde{\pi}^{\eta\eta}/\tau^2\frac{\tilde{u}^{\tau}}{\tau} - \left( \frac{2\tilde{u}^{\eta}\tilde{\pi}^{\tau\eta}}{\tau^3} +\frac{2\tilde{u}^{\tau}\tilde{\pi}^{\eta\eta}}{\tau^3} \right)
\end{eqnarray}

After simplification, 
\begin{eqnarray}
\frac{1}{\tau^2}\tilde{\partial}_{\lambda}(\tilde{u}^{\lambda}\tilde{\pi}^{\eta\eta}) =   - \frac{\tilde{\pi}^{\eta\eta}/\tau^2 - \tilde{\pi}_{NS}^{\eta\eta}/\tau^2}{\tau_{\pi}} - \frac{1}{3}\tilde{\pi}^{\eta\eta}\tilde{\theta}/\tau^2 - \tilde{g}_{\alpha\beta}(\tilde{u}^{\eta}\tilde{\pi}^{\eta\beta} + \tilde{u}^{\eta} \tilde{\pi}^{\eta\beta})\tilde{D}\tilde{u}^{\alpha}/\tau^2  + \tilde{\pi}^{\eta\eta}/\tau^2\frac{\tilde{u}^{\tau}}{\tau} - \frac{2\tilde{u}^{\eta}\tilde{\pi}^{\tau\eta}}{\tau^3} 
\end{eqnarray}



\end{document}  