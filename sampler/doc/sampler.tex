\documentclass[11pt, oneside]{article}   	% use "amsart" instead of "article" for AMSLaTeX format
\usepackage{geometry}                		% See geometry.pdf to learn the layout options. There are lots.
\geometry{letterpaper}                   		% ... or a4paper or a5paper or ... 
%\geometry{landscape}                		% Activate for rotated page geometry
%\usepackage[parfill]{parskip}    		% Activate to begin paragraphs with an empty line rather than an indent
\usepackage{graphicx}				% Use pdf, png, jpg, or eps§ with pdflatex; use eps in DVI mode
								% TeX will automatically convert eps --> pdf in pdflatex		
\usepackage{amssymb}

%SetFonts

%SetFonts


\title{Monte Carlo Sampler to get hadrons from freeze out hypersurface}
\author{LongGang Pang}
%\date{}							% Activate to display a given date or no date

\begin{document}
\maketitle
\section{Overview}

The number of particle species $i$ emmited from freeze out hypersurface
element $\Sigma^{\mu}$ is:

\[
dN_{i}=\frac{p\cdot\Sigma}{(2\pi)^{3}}\frac{d^{3}p}{p^{0}}f({\rm p})\Theta(p\cdot\Sigma)
\]


Where $\Theta(p\cdot\Sigma)$ is a heavyside step function that will
throw particles when $p\cdot\Sigma<0$. This function should work
for both $(t,x,y,z)$ coordinates and $(\tau,x,y,\eta_{s})$ coordinates,
except that the definitions of $p^{\mu}$ and $\Sigma^{\mu}$ are
different,

\begin{eqnarray*}
p^{\mu} & = & (E,p^{x},p^{y},p^{z})\ in\ (t,z)\\
p^{\mu} & = & (m_{T}\cosh(Y-\eta_{s}),p^{x},p^{y},m_{T}\sinh(Y-\eta_{s}))\ in\ (\tau,\eta_{s})
\end{eqnarray*}


The $f({\rm p})=\frac{1}{\exp((p\cdot u-\mu)/T)+\lambda}$ is the
distribution function which can be Bose-Einstein distribution for
mesons with $\lambda=-1$ , Fermi-Dirac distribution for baryons with
$\lambda=1$ or Juttner distribution with $\lambda=0$ when $\exp((p\cdot u-\mu)/T)\gg1$
for massive hadrons. There is big difference between Bose-Einstein
distribution and Juttner distribution for pion, while for other hadrons
the difference is quite negligible and we can use Juttner distribution
as an approximation of Bose-Einstein or Fermi-Dirac distribution for
all the hadrons except pion. 

In Local Rest Frame of $(t,x,y,z)$ coordinates, $u^{\mu}=(u^{t},u^{x},u^{y},u^{z})=(1,0,0,0)$
and $f({\rm p})=\exp\left(-(E-\mu)/T\right)$.

In Local Rest Frame of $(\tau,x,y,\eta_{s})$ coordinates, 
\[
u^{\mu}=(u^{\tau},u^{x},u^{y},u^{\eta})=\left(u^{t}\cosh\eta_{s}-u^{z}\sinh\eta_{s},u^{x},u^{y},-u^{t}\sinh\eta_{s}+u^{z}\cosh\eta_{s}\right)=(1,0,0,0)
\]
which means $v_{z}=\tanh\eta_{s}$ and the rapidity of fluid velocity
becomes $Y_{v}=\eta_{s}$ and the 4 momentum 
\[
p^{\mu}=(m_{T}\cosh(Y'+Y_{v}-\eta_{s}),p^{x},p^{y},m_{T}\sinh(Y'+Y_{v}-\eta_{s}))
\]
becomes $\tilde{p}^{\mu}=(m_{T}\cosh Y',p^{x},p^{y},m_{T}\sinh Y')=(E,p^{x},p^{y},p^{z})$
at local rest frame. 
The distribution function again becomes 
$f({\rm p})=\exp\left(-(p^{0}-\mu)/T\right)=\exp\left(-(m_{T}\cosh(Y)-\mu)/T\right)=\exp\left(-(E-\mu)/T\right)$.
It is not straight forward, but finally we see even in $(\tau,x,y,\eta_{s})$
coordinates, the distribution function in local rest frame equals
to thermal distribution in $(t,x,y,z)$ coordinates. 

$dN_{i}$ is Lorentz invariant quantity, so it is straight foward
to do the 3D momentum phase space integration in local rest frame
where the integration is anlytically equals to modified Bessel function
for isotime freeze out {[}H. Petersen{]} or simplified to 1D numerical
integration for arbitrary freeze out hyper surface {[}Hirano{]}. Once
we know the total number of hadrons $dN$ where $dN=\sum_{i}dN_{i}$,
we may do poisson sampling with $dN$ as the probability to determine
how many hadrons will actually be created. One may wonder whether
the results are different if we do poisson sampling for each species
with probability $dN_{i}$ (obviousely this is much slower). There
is a proof tells us that the results are equal to each other,

Theorem: Let $x_{1}$, $x_{2}$ are independent Poisson random variable
where $x_{i}$ has Poisson probability $\lambda_{i}$, then $x_{1}+x_{2}$
has a Poisson distribution with $\lambda_{1}+\lambda_{2}$.

\begin{eqnarray*}
\\
P(x_{1}+x_{2}=z) & = & f_{Z}(z)=\sum_{x=0}^{z}f_{x_{1}}(x)f_{x_{2}}(z-x)\\
 & = & \sum_{x=0}^{z}\frac{\lambda_{1}^{x}}{x!}e^{-\lambda_{1}}\frac{\lambda_{2}^{z-x}}{(z-x)!}e^{-\lambda_{2}}\\
 & = & e^{-(\lambda_{1}+\lambda_{2})}\sum_{x=0}^{z}\frac{\lambda_{1}^{x}}{x!}\frac{\lambda_{2}^{z-x}}{(z-x)!}\\
 & = & \frac{(\lambda_{1}+\lambda_{2})^{z}}{z!}e^{-(\lambda_{1}+\lambda_{2})}
\end{eqnarray*}


Where binormial formule has been used here $(a+b)^{z}=\sum_{x=0}^{z}C_{z}^{x}a^{x}b^{z-x}=\sum_{x=0}^{z}\frac{z!}{x!(z-x)!}a^{x}b^{z-x}$.

After the real number of hadrons is determined, the next step is to
determine the particle type, which can be done by using the discrete
distribution with probabilities given by $dN_{i}$. And there is sampling
functions in c++11 for poisson distribution and discrete distribution
which can be used directly. 


\section{Mometum sampling}

It is easy to sample the 4 momentum from distribution function $f(p)=p^{2}\exp(-\sqrt{p^{2}+m^{2}}/T)$
by rejection method.This is a good idea. Scott Pratt introduces a
smart way to sample 4 momentum from thermal distributions, the math
trick is used here: for probability distribution $x^{n-1}e^{-x}$,
one can sample the distribution by taking the natural log of $n$
random numbers $x=-\ln(r_{1}r_{2}...r_{n})$ where $r_{i}$ are random
numbers uniformly distributed between zero and one. For three dimentional
thermal distribution for massless particle whose distribution function
reads $f(p)=p^{2}e^{-p/T}$, 

\begin{eqnarray*}
p & = & -T\ln(r_{1}r_{2}r_{3})\\
\cos\theta & = & \frac{\ln(r_{1})-\ln(r_{2})}{\ln(r_{1})+\ln(r_{2})}\\
\phi & = & \frac{2\pi\left[\ln(r_{1}r_{2})\right]^{2}}{\left[\ln(r_{1}r_{2}r_{3})\right]^{2}}
\end{eqnarray*}


By checking the Jacobian, indeed

\begin{eqnarray*}
dpd\cos\theta d\phi & = & |J|\ dr_{1}dr_{2}dr_{3}\\
 & = & \frac{8\pi T}{r_{1}r_{2}r_{3}\left[\ln(r_{1}r_{2}r_{3})\right]^{2}}dr_{1}dr_{2}dr_{3}\\
 & = & \frac{8\pi T}{e^{-p/T}p^{2}/T^{2}}dr_{1}dr_{2}dr_{3}
\end{eqnarray*}


And $dr_{1}dr_{2}dr_{3}=\frac{1}{8\pi T^{3}}p^{2}e^{-p/T}dpd\cos\theta d\phi$.

For massive hadrons with Juttner distribution,

\[
p^{2}e^{-(E-\mu)/T}=p^{2}e^{-p/T}e^{(p-E+\mu)/T}
\]


Use acceptance-rejection method, draw $p$ from $p^{2}e^{-p/T}$ distribution,
accept or reject with weight function $\omega(p)=e^{(p-E)/T}=e^{(p-\sqrt{p^{2}+m^{2}})/T}$.
Notice that $\omega(p)<1$ is always satisfied, while for heavy hadrons
$\omega(p)\ll1$ and many samplings are rejected which makes this
method inefficient.


\subsection*{For $\pi$}

\[
p^{2}/(e^{(E-\mu)/T}-1)=p^{2}e^{-p/T}*e^{p/T}/(e^{(E-\mu)/T}-1)
\]


Use acceptance-rejection method, draw $p$ from $p^{2}e^{-p/T}$ distribution,
accept or rejection with weight function $\omega(p)=e^{p/T}/(e^{(E-\mu)/T}-1)$.
However in this case we don't know when $\omega(p)<1$ will break
down. Do a simple test with $mass=0.139$, $T=0.150$ shows that $\omega(p)<1$
will break down soon at large $p$ for $\mu=0$ and every where for
non-zero $\mu$. This is frustrating because most of the hadrons produced
are pions and we already know that there are big difference for Juttner
distribution and Bose-Einstein distribution for pion. However if we
know one simple relationship betwen the maximum value of $\omega(p)$
and $p,\mu$, we can scale the weight function by $\omega(p)/\omega_{max}(\mu)$
to fix it. For big $\mu$ the probability for rejection is around
$50\%$ which is still acceptable as shown in Fig 2. One remaining
question is to find the maximum value of $\omega(p)$, as you can
see there are $2$ points with $\frac{d\omega(p)}{dp}=0$ for $\mu\neq0$
and one can not determine if it is maximum. For $\mu=0$, non of these
2 points around $p=0.1$ produce maximum since $\omega(p)$ may exceed
unity at large $p$. So further check should be done if $\omega_{max}(p\sim0.1)>\omega(p_{max})$
where $p_{max}$ is the biggest momentum cut that will be used. Another
way to keep $\omega(p)<1$ is to rescale it by $\exp(a*\mu/T)$ however
the optimised value of $a$ is unknow.

For heavy hadrons there is no difference between BoseEinstein, FermiDirac
and Juttner distribution, so Juttner distribution can be used to sample
the 4 momentum of all hadrons except $\pi.$ The weight function $\omega(p)=e^{(p-E)/T}$
is much smaller for heavier hadrons, so there are many rejections
which will make the program slow. However, The philosophy is for $T/m<0.6$
do variable transformation:


\begin{eqnarray}
p & = & \sqrt{E^{2}-m^{2}},\ dp=E/pdE\\
dpp^{2}e^{-E/T} & = & dE\frac{E}{p}p^{2}e^{-E/T}\\
 & = & dEpEe^{-E/T}\\
 & = & dk\frac{p}{E}(k+m)^{2}e^{-k/T}e^{-m/T}\\
 & = & dk(k+m)^{2}e^{-k/T}\omega(p)\\
 & = & dk(k^{2}+2mk+m^{2})e^{-k/T}\omega(p)
\end{eqnarray}


where $k=E-m$ and $\omega(p)=\frac{p}{E}e^{-m/T}$ is weight function.
The excelent part of this algorithm is that $E-m>0$ and $p/E<1$.
The $e^{-m/T}$ and $e^{-\mu/T}$ terms are not important and can
be dropped. By split the upper distributions into 3 parts and determine
which part is domainant by their integrated weight.

\begin{eqnarray}
\int dkk^{2}e^{-k/T} & = & 2T^{3}\\
\int dk2mke^{-k/T} & = & 2mT^{2}\\
\int dkm^{2}e^{-k/T} & = & m^{2}T
\end{eqnarray}


Once picked k one can do rejection and repeat with weight $p/E$. 

\section{Adaptive Rjection Sampling}

Since the Bose-Einstein distribution for pion is quite special, and
the 4-momentum is not easy to sample with finite chemical potential
$\mu$ where weight function $\omega(p)<1$ is not satisfied, it is
possible to further rescale $\omega(p)$ so that it is smaller than
unity. But the best scale factor is arbitrary with unkown parameter
$a$. It is possible to use Adaptive Rejection Sampling (ARS) which
will construct one upper bound and refine this bound with rejected
points. ARS asks for the probability distribution function $f(x)$
to be log concave, which means if we set $h(x)=\log f(x)$, $h''(x)<0$
should be true for any $x$, fortunatelly all the Juttner, Bose-Einstein
and Fermi-Dirac distributions obey this rule. The ARS method is developed
further to isolate the distribution function to concave and convex
part which will produce upper bounds separately. 

The philosophy of ARS is to generate a piecewise exponential distribution
upper bound for $f(x)$. Where the distribution function $q(x)$ is
piecewise exponential distribution $\mbox{q(x)\ensuremath{\propto\exp}(g(x))}$if
$g(x)$ is piecewise linear. The ordered change points are $z_{0}<z_{1}<z_{2}...<z_{n}$
and $g(x)$ has slope $m_{i}$ in $(z_{i-1},z_{i})$. The area under
each piece of exponential segment $\exp(g(x_{i}))$ is,

\[
A_{i}=\int_{z_{i}-1}^{z_{i}}e^{g(x)}dx=\frac{1}{m_{i}}\left(e^{g(z_{i})}-e^{g(z_{i-1})}\right)
\]


First sample $j$ from discrete\_distribution(\{$A_{i}$\}), then
sampling $x\in(z_{j-1,}z_{j})$ from distribution function $q(x)=\exp(a+m_{i}x)$.
By inversely sampling uniform distribution $r\in[0,1]$ from the cumulative
propability 

\[
Q(x)=\int_{z_{i-1}}^{x}q(y)dy=\frac{q(x)-q(z_{i-1})}{q(z_{i})-q(z_{i-1})}=r
\]


we get $x$ from the exponential distribution,

\[
x=\frac{1}{m_{i}}\ln\left(re^{m_{i}z_{i}}+(1-r)e^{m_{i}z_{i-1}}\right)
\]


With this $x$ we can do rejection test: $ran()<\frac{f(x)}{q(x)}=\exp(h(x)-g(x))$.

If a point is rejected, it will be used to refine the upper bound
which will make the upper bound close to $f(x)$ as soon as possible.
In squeezing test step, lower bound is also needed which we call $l(x)$.
Squeezing test is true if $ran()<\frac{l(x)}{q(x)}$. 

There are 3 kinds of different upper bounds, one is from tangent lines,
the other two are from scants, the upper bounds given by scants is
looser but there will less derivtive calculations.



\section{(3+1)D viscous hydrodynamics hydrodynamics}


For ideal hydrodynamics, sample $n d\Sigma_{max}$ number of hadrons, determine the particle type using discrete distribution
with the probabilities given by their equilibrium density $n_i$, where $n=\sum_i n_i$.

Once the particle type is determine, we sample the four-momenta of the particle in the local rest frame of the fluid cell,
and keep the particle if $r<\frac{p^*\cdot d\Sigma^*}{p^{0*}d\Sigma_{max}}$.

For viscous hydrodynamics, one needs to consider the non-equilibrium contributions. However, the $\pi^{\mu\nu}$ terms
does not contribute to the total number since the particle flow on freeze out hypersurface $n^{\mu} = n u^{\mu}$ is perpendicular
to $\pi^{\mu \nu}$. This property is really helpful since we don't need to calculate equilibrium density for each freeze out hyper-surface.

And we can sample N hadrons, where N equals to,
\[
N = n d\Sigma_{max} K_{max}
\]
where the non-equilibrium scale factor $K_{max}$ is used to get more hadrons for the additional rejection procedure,
\[
K_{max} = \left[ 1 + (1 \mp f_0) \frac{p_{\mu}^*p_{\nu}^*\pi^{\mu\nu*}}{2T^2(\varepsilon+P)}\right]_{max}
\]
After $N$ is determine, sample how many particles will be produced with Poisson distribution whose mean probability is $N$.
Then determine the particle type as before, sample four-momenta in LRF, do keep or rejection according to,
\[
r < \frac{p^*\cdot d\Sigma}{p^{0*}d\Sigma_{max}}  w_{visc} 
\]
where the rejection weight $w_{visc}$ from non-equilibrium corrections is,
\[
w_{visc} = \frac{A + (1\mp f_0)p_{\mu}^*p_{\nu}^*\pi^{\mu\nu*}}{\left[ A + (1\mp f_0)p_{\mu}^*p_{\nu}^*\pi^{\mu\nu*}\right]_{max}}
\]
where $A=2T^2(\epsilon+P)$  is constant on the freeze out hyper-surface.

Then for Fermions, 
\begin{eqnarray}
K_{max} &=& 1 + \left[ p_{\mu}^*p_{\nu}^*\pi^{\mu\nu*} \right]_{max}/A \\
w_{visc} &=& \frac{A + (1 - f_0)p_{\mu}^*p_{\nu}^*\pi^{\mu\nu*}}{A + \left[ p_{\mu}^*p_{\nu}^*\pi^{\mu\nu*} \right]_{max}}
\end{eqnarray}
and for Bosons,
\begin{eqnarray}
K_{max} &=& 1 + 2 \left[ p_{\mu}^*p_{\nu}^*\pi^{\mu\nu*} \right]_{max}/A \\
w_{visc} &=& \frac{A + (1 + f_0)p_{\mu}^*p_{\nu}^*\pi^{\mu\nu*}}{A + 2 \left[ p_{\mu}^*p_{\nu}^*\pi^{\mu\nu*} \right]_{max}}
\end{eqnarray}

The easiest way to get $\left[ p_{\mu}^*p_{\nu}^*\pi^{\mu\nu*} \right]_{max}$ is as follows,
\begin{eqnarray}
 p_{\mu}^*p_{\nu}^*\pi^{\mu\nu*}  & \le & \sum_{\mu \nu} | p^{\mu*} p^{\nu*} \pi^{\mu\nu*} |  \\
   & \le & E^{*2} \sum_{\mu \nu}  | \pi^{\mu\nu*} |  
\end{eqnarray}
thus we have $\left[ p_{\mu}^*p_{\nu}^*\pi^{\mu\nu*} \right]_{max} = E_{max}^{*2} \sum_{\mu \nu}  | \pi^{\mu\nu*}|$.

Now the only difficulty is how to boost $\pi^{\mu\nu}$ tensor to the comoving frame of the fluid.
\end{document}  
